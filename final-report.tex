\documentclass[a4j]{jarticle}

\textwidth=16cm

\oddsidemargin=0cm

\title{情報工学実験C ネットワークプログラミング}

\author{氏名:大西 隼也 \\学籍番号:09427510}

\date{出題日: 2017年12月12日\\提出日:2018年1月30日 \\締切日:2018年1月30日}

\begin{document}

\maketitle

\section{概要}
本実験では,基本的な通信方式であるTCP/IP,UDP/IPによるネットワークプログラミングについて学習する.
また,分散システムの基本的な形式であるクライアントサーバモデルの仕組みを学習する.
最終的に,クライアントサーバモデルに基づくプログラムを作成する.

\section{クライアント・サーバモデルの通信の仕組みについて}

\subsection{プロセス間通信の場合}

ソケットとはファイルの入出力と同様の処理でプロセス間通信を実現するためのもの

通信手順

通信相手プロセスとの間にソケットを生成する

ソケット番号(ファイルディスクリプタの番号)が返る

ソケットに対してsendやrecvを実行することでデータの送受信を実現する(ソケット番号で指定)

通信が終了したらソケットを削除する

\subsection{インターネットでの通信の仕組み}

計算機の識別方法は,インターネット上のすべての計算機には,一意のIPアドレスが割り振られている(IP:Internet Protocol)

同じIPアドレスを複数の計算機が持つことはない,(IPアドレスで計算機を特定可能)

\subsection{DNS(Domain Name Service)}

インターネット上のホスト名とIPアドレスを対応させるシステム

IPアドレスでは覚えにくく直感的にわかりにくい

ホスト名の例:www.okayama-u.ac.jp

ホスト名からIPアドレスに(正引き),IPアドレスからホスト名に(逆引き)変換可能

DNSサーバに処理を依頼すると,変換結果が得られる(nslookupコマンド)


\subsection{ポート番号}

同時に複数の計算機と通信することが考えられる

通信相手計算機に複数のプログラムが存在する場合どうやって相手を識別するのか?

→捕助アドレスとしてポート番号を利用する

ポート番号:0-65535の間で指定可能

サービス種別を判別するために用いられる

IPアドレスとポート番号で通信相手プログラムを指定

150.46.30.130と80番ポート:岡山大学のwebサーバ

well-knownポート

1023番までのポートは主要なプロトコルで用いられる番号が決まっている

FTP(20,21),SMTP(25),DNS(53),HTTP(80),POP3(110)


\subsection{クライアントサーバモデル}

クライアント:サービスを受けるプロセス

サーバに処理を要求,受け取った結果を利用して処理を行う

サーバ:サービスを提供するプロセス

クライアントから要求された処理を行い,結果を返す


\subsection{メッセージの内容}
実際にクライアントとサーバはどのようなメッセージをやりとりしているのか

プロトコル:通信規約

ネットワークを介してコンピュータ同士が通信を行う上で相互に決められた約束事の集合

RFC:インターネットの標準技術を定めた文書にて記されている


webクライアントはサーバに要求メッセージを送信し,応答メッセージを受信することを繰り返して,処理を行っている.

\subsection{クライアントの処理の流れ}
\begin{enumerate}
\item 通信相手のIPアドレスを取得
\item ソケットの作成
\item 接続の確立
\item 要求メッセージを送信
\item 応答メッセージを受信
\item 応答メッセージを処理
\item ソケットの削除
\end{enumerate}

\subsection{TCP/IPでの送受信関数}
主にクライアントで使う物を書く
\begin{itemize}
\item gethostbyname:IPアドレスを得る
\item socket:ソケットを作成する
\item connect:コネクションを確立させる
\item send:メッセージを送信する
\item recv:メッセージを受信する
\item close:ソケットを削除する
\end{itemize}

主にサーバで利用される物は,
\begin{itemize}
\item socket:ソケットを作成する
\item bind:ソケットに名前をつける
\item listen:接続要求を待つ
\item accept:接続要求を受け付ける
\item send:メッセージを送信する
\item recv:メッセージを受信する
\end{itemize}


\subsection{サーバプログラムの処理}
サーバは要求メッセージの到着を常に待ち,要求メッセージが到着したら処理を行い,結果を送信する.
\begin{itemize}
\item ソケットの作成
\item ソケットに名前をつける
\item 接続要求の受付を開始する
\item 接続要求を受け付ける
\item 要求メッセージを受信
\item 要求メッセージを処理
\item 応答メッセージを送信する
\item 次の接続要求の受付を開始する
\end{itemize}


\section{名簿管理プログラムのクライアント・サーバプログラムの作成方針}

\subsection{名簿管理プログラムの仕様について}
基本的にはプログラミング演習で作成した名簿管理プログラムの入出力部分をsend関数やrecv関数を用いて書き換えを行い,サーバ,クライアント間で通信を行えるように実装し直す.

\section{プログラム及び,その説明}


\subsection{TCP/IPのプロトコルの説明}
TCP:Transmissiomn Control Protocol

IP:Internet Protocol

IPは,データグラムを転送するためのプロトコルであり,アドレスとしてIPアドレスとポート番号を用いる.

TCPは,ストリーム転送サービスを提供しており,これにより,信頼性と複数回に分けて送り出したデータについても,順序を保証することが可能になっている.しかし,その分通信に時間がかかるというデメリットもある.



\section{プログラムの使用法}

\section{プログラムの作成過程に関する考察}

\subsection{工夫した点}
\subsubsection{ソケットの再送待機状態対策}

\subsection{作成に苦労した点}
\subsubsection{\%Pコマンドの実装について}

\section{得られた結果に関する考察}

\section{作成したプログラム}



\end{document}	


