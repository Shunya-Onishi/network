\documentclass[a4j]{jarticle}

\usepackage{url}

\textwidth=16cm

\oddsidemargin=0cm

\title{情報工学実験C ネットワークプログラミング}

\author{氏名:大西 隼也 \\学籍番号:09427510}

\date{出題日: 2017年12月12日\\提出日:2018年1月30日 \\締切日:2018年1月30日}

\begin{document}

\maketitle

\section{概要}
本実験では,基本的な通信方式であるTCP/IP,UDP/IPによるネットワークプログラミングについて学習する.
また,分散システムの基本的な形式であるクライアントサーバモデルの仕組みを学習する.
最終的に,クライアントサーバモデルに基づくプログラムを作成する.

\section{クライアント・サーバモデルの通信の仕組みについて}
クライアントサーバモデルとは,クライアントとサーバを分離して管理するソフトウェアモデルであり,今回の実験ではローカルの環境下で動作するプログラムを作成したが,ここでは,代表的なサーバモデルである,メールサーバやwebサーバとクライアントがインターネットを通じて通信する際の方法について概要を説明した後,実装したプログラムのソケットを用いた通信について述べる.

\subsection{インターネットでの通信の仕組み}

インターネット上のすべての計算機には,一意のIPアドレスが割り振られているため,IPアドレスを用いて計算機を特定することができる.しかし実際にIPアドレスを用いて通信を行おうとすると,例えば岡山大学のIPアドレス(150.46.30.130)など数字の羅列で人間が直感的に分かりにくいため,インターネット上のホスト名(www.okayama-u.ac.jp)とIPアドレスを対応させるシステム,DNS(Domain Name Service)が用いられている.

DNSもサーバの一種であり,ターミナル上でnslookupコマンドなどを用いて処理を依頼すると,ホスト名からIPアドレスに(正引き),IPアドレスからホスト名に(逆引き)の変換結果を返す.

また同時に複数の計算機と通信する際や通信相手計算機に複数のプログラムが存在する場合には,IPアドレスに加えて補助アドレスとしてポート番号を利用する.

ポート番号とは,0-65535の間で指定可能な数であり,サービスの種別を判断するために用いられる.
例えば,IPアドレス(150.46.30.130)とポート番号(80)は岡山大学のwebサーバを示す.

自作でプログラムを作成する際に注意したいのは,1023番までのポートはwell-knownポートと呼ばれ,主要なプロトコルで用いられる番号が決まっているため,それ以外の番号を利用する必要が有る.

\begin{itemize}
\item DNS(53),HTTP(80),POP3(110)など
\end{itemize}


\subsection{今回作成したクライアントサーバモデルでの通信}
以上述べてきたように,インターネットはTCP/IPという通信プロトコルを用いて通信を行っているが,実際にTCP/IPをプログラムから利用するには,プログラムとインターネットをつなぐ出入り口が必要になってくる.その出入り口となるのがソケットと呼ばれるものであり,TCP/IP通信はソケット通信と呼ばれることもある.

ソケットの最大の特徴として,ソケットを介してデータを送受信する際の要領が基本的にファイル入出力と同じであり,扱いやすいという利点がある.そのため単純なプロセス間通信では,通信相手プロセスとの間にソケットを生成し,そのソケット番号を通信に利用しながらソケットに対して送信や受信の命令を実行することでデータの送受信を実現している.


\section{名簿管理プログラムのクライアント・サーバプログラムの作成方針}

\subsection{名簿管理プログラムの仕様について}
基本的にはプログラミング演習で作成した名簿管理プログラムの入出力部分をsend関数やrecv関数を用いて書き換えを行い,サーバ,クライアント間で通信を行えるように実装し直す.

今回与えられた仕様として,名簿管理プログラム終了時,クライアントのみ終了し,サーバ側のプログラムは接続待機状態に戻り待つというものがあったので,通信部をwhile文で意図的に無限ループするような方針でプログラムを作成した.

\section{プログラム及び,その説明}

\subsection{TCP/IPのプロトコルの説明}

IPは,Internet Protocolの略称で,データグラムを転送するためのプロトコルとされている,アドレスとしてIPアドレスとポート番号を用いる.

TCPは,Transmissiomn Control Protocolの略称で,ストリーム転送サービスを提供しており,これにより,信頼性と複数回に分けて送り出したデータについても,順序を保証することが可能になっている.しかし,その分通信に時間がかかるというデメリットもある.

\subsection{名簿管理プログラムのコマンド一覧}

      \begin{tabular}{|l|l|l|}
        \hline
        コマンド& 意味& 備考\\
        \hline
        \%Q & 終了(Quit)& \\
        \hline
        \%C & 登録件数などの表示(Check)& \\
        \hline
        \%P n& 先頭からn件表示(Print)& n$=$0: 全件表示, n $<$ 0 後ろから -n 件表示\\
        \hline
        \%R file& fileから読み込み(Read)& \\
        \hline
        \%W file & fileへ書き出し(Write)& \\
        \hline
        \%F word& wordを検索(Find)& 結果を\%Pと同じ形式で表示 \\
        \hline
        \%S n& データをn番目の項目で整列(Sort)& 表示はしない\\
        \hline
        \%D n& データをn件削除(Delete)& 仕様は後述する \\
        \hline
        \%A n& n番目にデータを登録(Add)& \\
        \hline
        \%B & 直前の状態に戻る(Back)& \%R,\%Aの使用後のみ\\
        \hline
        \%M & 各コマンドの仕様を表示(Manual)& \\
        \hline
      \end{tabular}
      
      \subsection{クライアントの処理の流れ}
ここでは,クライアントプログラムの主な処理の流れと, 今回使用したTCP/IPの関数について述べる.
\begin{enumerate}
\item 通信相手のIPアドレスを取得
	\begin{itemize}
	\item gethostbyname:IPアドレスを得る
	\end{itemize}
\item ソケットの作成
	\begin{itemize}
	\item socket:ソケットを作成する
	\end{itemize}
\item 接続の確立
	\begin{itemize}
	\item connect:コネクションを確立させる
	\end{itemize}
\item 要求メッセージを送信
	\begin{itemize}
	\item send:メッセージを送信する
	\end{itemize}
\item 応答メッセージを受信
	\begin{itemize}
	\item recv:メッセージを受信する
	\end{itemize}
\item 応答メッセージを処理
\item ソケットの削除
	\begin{itemize}
	\item close:ソケットを削除する
	\end{itemize}
\end{enumerate}

\subsection{サーバの処理の流れ}
サーバは要求メッセージの到着を常に待ち,要求メッセージが到着したら処理を行い,結果を送信する.
ここでは,サーバプログラムの主な処理の流れと,今回使用したTCP/IPの関数について述べる.
\begin{enumerate}
\item ソケットの作成
	\begin{itemize}
	\item socket:ソケットを作成する
	\end{itemize}
\item ソケットに名前をつける
	\begin{itemize}
	\item bind:ソケットに名前をつける
	\end{itemize}
\item 接続要求の受付を開始する
	\begin{itemize}
	\item listen:接続要求を待つ
	\end{itemize}
\item 接続要求を受け付ける
	\begin{itemize}
	\item accept:接続要求を受け付ける
	\end{itemize}
\item 要求メッセージを受信
	\begin{itemize}
	\item recv:メッセージを受信する
	\end{itemize}
\item 要求メッセージを処理
\item 応答メッセージを送信する
	\begin{itemize}
	\item send:メッセージを送信する
	\end{itemize}
\item 次の接続要求の受付を開始する
\end{enumerate}

\section{プログラムの使用法}
\subsection{クライアントプログラムの動作}
クライアント側のプログラムは起動後,ソケットを作成し,サーバ側との通信を確立した後,コマンドの入力待ちを行う.コマンドが入力されるとソケットを介してコマンドをサーバ側に送信し,処理結果を受け取り表示を行う.
\begin{verbatim}
oonishishunya-no-MacBook-Air:network oonishishunya$ ./meibomac-client localhost
クライアントの入力待ち
%C
登録件数は0件です.

クライアントの入力待ち
%R sample.csv
読み込みが完了しました.%C等で確認してください.

クライアントの入力待ち
%C
登録件数は2886件です.

クライアントの入力待ち
%Q
終了します。
\end{verbatim}

\subsection{サーバプログラムの動作}
サーバ側のプログラムは起動後,ソケットを作成し通信の準備が整ったらクライアントからの入力を待ち,\%Q以外のコマンドを受け取った際には処理を開始し結果をクライアントに返す.

\%Qコマンドを受け取った際には,処理を行わず通信待機状態へと戻る.
\begin{verbatim}
oonishishunya-no-MacBook-Air:network oonishishunya$ ./meibomac-server 
クライアントの入力待ち 
入力 %C
サーバの処理開始 
入力 after parse_line(): %C
処理終了

クライアントの入力待ち 
入力 %R sample.csv
サーバの処理開始 
入力 after parse_line(): %R sample.csv
処理終了

クライアントの入力待ち 
入力 %C
サーバの処理開始 
入力 after parse_line(): %C
処理終了

クライアントの入力待ち 
入力 %Q
処理終了
\end{verbatim}
\section{プログラムの作成過程に関する考察}

\subsection{工夫した点}
\subsubsection{ソケットの再送待機状態対策}
ソケットの特性上,一度ソケットをクローズすると,最初にクローズした側(今回の場合はクライアント側)は,再送待機状態(TIME\_WAIT)になる.この状態では,プログラム動作外で到着したパケットを破棄できるよう,数分間はCLOSED状態にならないので,他のソケットがそのポートを使用することができない.
対策法としては,最も単純なのがプログラムを起動するたびに使用するポートを変えるなどが考えられるが,今回は作成したソケットにSO\_REUSEADDRオプションを付加することで問題を解決した.

SO\_REUSEADDRオプションを付加することで,同じローカルアドレスにbindを行ってもエラーにならず処理を行うことができる.

\begin{verbatim}
  /* SO_REUSEADDR をつける*/

  int ret;
  ret = setsockopt(sockfd, SOL_SOCKET, SO_REUSEADDR,(const char *)&yes, sizeof(yes));
  if(ret < 0){
    printf("Error : can't opt");
    return 0;
  }
\end{verbatim}

\subsection{作成に苦労した点}

\subsubsection{\%Pコマンドの実装について}
\%Pコマンドに代表されるように,サーバ側のプログラムから複数件の結果が返される際,パケットの分割サイズを予測することができないため,\%Pコマンドの出力件数が一つずれたり,sendとrecvのデータ送受信の関係からコマンド実行後,意図しない出力が行われコアダンプを起こすなどの問題が頻発した.

そこで,\%Pコマンドを入力した際には,実際にプリント処理を行う,recv,sendのとは別に,何回プリント処理を行うか回数についてのみのデータの送受信を行うよう実装を行い,その後そのループ回数だけrecvを行うようクライアントプログラムを記述することで,問題を解決することができた.

\begin{verbatim}
    /*メッセージを受信する*/
    char kekka[MAX_LINE_LEN + 1];
    if((line[0]=='%' && line[1]=='P') || (line[0]=='%' && line[1]=='F')){
      bzero(&kekka, sizeof(kekka));
      recv(sockfd, kekka, sizeof(kekka), 0); //回数を受け取る
      int times;
      int l;
      times = atoi(kekka);
      for(l=0; l<times; l++){
	bzero(&kekka, sizeof(kekka));
	recv(sockfd, kekka, sizeof(kekka), 0);
	printf("%s\n", kekka);
      }
\end{verbatim}

\section{得られた結果に関する考察}
\subsection{recvとsendの対応付けに関する問題}
今回の実験は,クライアントサーバモデルを構成して以前作成したプログラムを動かすというものだったので,実際の名簿管理プログラムの速度的な性能や機能面などの考察はテーマに沿わないため,6章の作成過程に関する考察が中心となった.ここでは,プロセス間通信で想定外の動作を防ぐため特に注意し,実装に手間取ったrecvとsendの対応付けに関する問題について考察する.

自作の名簿管理プログラムのコマンドの中で,recvとsendの対応付けが一対一対応にならないものは,
\%P,\%F,\%A,\%Bの4つであった.

\%Pや\%Fの場合,問題になるのはプリント回数のみであるので,前述の考察で実装したように,プリント回数とプリント処理部をわけて通信を行うといった対応が可能だったが,\%Aや\%Bなど自作で機能拡張したコマンドなどはその関数の実装法などが一般的ではない場合もあるため,今回はクライアント側ですべて場合分けをしてコマンドごとにrecvとsendの回数を合わせにいったが,現実的ではないことがわかる.

インターネットやメールサーバなどユーザ数や同時アクセスの問題を考えると,個々の処理はその関数の中で完結するように実装を行い,ユーザインタフェースにあたる,クライアント部などのプログラムは単純に記述を行うことが,後々の仕様変更や保守の観点から見れば重要であると考えられる.

\subsection{考察に付随した感想}
自分の作成したプログラムでも1年前ともなると何を書いているか,関数がどう動いているかわからない部分が多く,コメントやレポートを丁寧に書いておくことの重要性を実感した.
また,先ほど考察したように,理想的には,名簿管理プログラムの内部で処理を完結させるような構造にしたかったが,当時プログラムが動けばさえすればいいと思ってコードを書いていた部分も多くあり,今回のように後から変更が加えづらい点でとても苦労したので,今後は保守性も意識してプログラムを作成しようと感じた.


\section{作成したプログラム}
今回,作成したプログラムのソースコードについて,名簿管理の処理を行うメインのプログラムに加え,その通信部分を担うクライアント,サーバプログラムの3つに分かれており,非常に膨大なページ数となるためgithubへのリンクと,プログラム名を記載することで割愛する.

\url{https://github.com/Shunya-Onishi/network}

\begin{itemize}
\item 名簿管理プログラム本体:meibo-prog.c p8-
\item 名簿管理プログラムクライアント:meibo-client.c p22-
\item 名簿管理プログラムサーバ:meibo-server.c p26-
\end{itemize}

\newpage

\subsection{追記:ソースコード}
\subsubsection{meibo-prog.c}
\begin{verbatim}
     1	/*[1]*/
     2	#include <stdio.h>
     3	#include <stdlib.h>
     4	#include <string.h>
     5	#include <sys/types.h>
     6	#include <sys/socket.h>
     7	
     8	
     9	#define MAX_LINE_LEN 1024
    10	#define MAXSTR 69
    11	#define MAXPRO 10000
    12	#define MAX_ID_LEN 31
    13	#define MAX_BIRTH_LEN 10
    14	
    15	
    16	int back = 0;
    17	int ditems;
    18	int mark = 0;
    19	int flag =0;
    20	
    21	/*[2]*/
    22	struct date {
    23	  int y;
    24	  int m;
    25	  int d;
    26	};
    27	
    28	/*[3]*/
    29	struct profile {
    30	  int id;
    31	  char name[MAXSTR+1];
    32	  struct date birthday;
    33	  char home[MAXSTR+1];
    34	  char *comment;
    35	};
    36	
    37	/*[4]*/
    38	struct profile profile_data_store[MAXPRO];
    39	int profile_data_nitems = 0; 
    40	
    41	void parse_line(char *line, int new_s);
    42	
    43	/*[5]*/
    44	int subst(char *str, char c1, char c2)
    45	{
    46	  int n = 0;
    47	
    48	  while (*str) {
    49	    if (*str == c1) {
    50	      *str = c2;
    51	      n++;
    52	    }
    53	    str++;
    54	  }
    55	  return n;
    56	}
    57	
    58	/*[6]*/
    59	int split(char *str, char *ret[], char sep, int max)
    60	{
    61	  int cnt = 0;
    62	
    63	  ret[cnt++] = str;
    64	
    65	  while (*str && cnt < max) {
    66	    if (*str == sep){
    67	      *str = '\0';
    68	      ret[cnt++] = str + 1;
    69	    }
    70	    str++;
    71	  }
    72	 return cnt;
    73	}
    74	
    75	/*[7]*/
    76	int get_line(FILE *fp,char *line)
    77	{
    78	  if (fgets(line, MAX_LINE_LEN + 1, fp) == NULL)
    79	    return 0;
    80	
    81	  subst(line, '\n','\0');
    82	
    83	  return 1;
    84	}
    85	
    86	/*[8]*/
    87	struct date *new_date(struct date *d, char *str)
    88	{
    89	  char *ptr[3];
    90	
    91	  if (split(str, ptr, '-', 3) != 3)
    92	    return NULL;
    93	
    94	  d->y = atoi(ptr[0]);
    95	  d->m = atoi(ptr[1]);
    96	  d->d = atoi(ptr[2]);
    97	
    98	  return d;
    99	}
   100	
   101	/*[9]*/
   102	struct profile *new_profile(struct profile *p, char *csv){
   103	  char *ptr[5];
   104	
   105	  if (split(csv, ptr, ',', 5) != 5)
   106	    return NULL;
   107	
   108	  p->id = atoi(ptr[0]); 
   109	
   110	  strncpy(p->name, ptr[1], MAXSTR);
   111	  p->name[MAXSTR] = '\0';
   112	
   113	  if (new_date(&p->birthday, ptr[2]) == 0)
   114	    return 0;
   115	
   116	  strncpy(p->home, ptr[3], MAXSTR);
   117	  p->home[MAXSTR] = '\0';
   118	
   119	  p->comment = (char *)malloc(sizeof(char) * (strlen(ptr[4]) +1));
   120	  strcpy(p->comment, ptr[4]);
   121	
   122	  flag = 1;
   123	  return p;
   124	}
   125	
   126	/*[10]*/
   127	void cmd_quit(char *param, int new_s)
   128	{
   129	  char s[MAX_LINE_LEN + 1]={'\0'};
   130	  if(flag==0){
   131	    snprintf(s, MAX_LINE_LEN, "終了します。\n");
   132	    send(new_s, s, sizeof(s), 0);
   133	    // exit(0);
   134	      }
   135	  if(*param == 'a'){
   136	    snprintf(s, MAX_LINE_LEN, "終了します。\n");
   137	    send(new_s, s, sizeof(s), 0);
   138	    // exit(0);
   139	      }
   140	
   141	  else {
   142	    //    snprintf(s, MAX_LINE_LEN,"入力されたデータが保存されていません.\n%%Q a でこのまま終了します.\n");
   143	    snprintf(s, MAX_LINE_LEN, "終了します。\n");
   144	    send(new_s, s, sizeof(s), 0);
   145	  }
   146	}
   147	/*[11]*/
   148	void cmd_check(int new_s)
   149	{
   150	  //printf("登録件数は%d件です.\n", profile_data_nitems);
   151	  char s[MAX_LINE_LEN + 1] = {'\0'};
   152	  snprintf(s, MAX_LINE_LEN, "登録件数は%d件です.\n", profile_data_nitems);
   153	  send(new_s, s, sizeof(s), 0);
   154	}
   155	/*[12]*/
   156	char *date_to_string(char buf[], struct date *date)
   157	{
   158	  sprintf(buf, "%04d-%02d-%02d", date->y, date->m, date->d); /*文字列の中に入れる*/
   159	  return buf;
   160	}
   161	
   162	/*[13]*/
   163	void print_profile(struct profile *p, int new_s)
   164	{
   165	  char date[11];
   166	  char s[MAX_LINE_LEN+1] = {'\0'};
   167	
   168	  snprintf(s, MAX_LINE_LEN,"Id    : %d\nName  : %s\nBirth : %s\nAddr  : %s\nCom.  : %s\n", 
   169		   p->id, p->name, date_to_string(date, &p->birthday), p->home, p->comment);
   170	  send(new_s, s, sizeof(s), 0);
   171	
   172	}
   173	
   174	/*[14]*/
   175	void cmd_print(int nitems, int new_s)
   176	{
   177	  int i, end = profile_data_nitems;
   178	  char s[MAX_LINE_LEN +1]={'\0'};
   179	
   180	  if(nitems == 0){
   181	    snprintf(s, MAX_LINE_LEN, "%d", end);
   182	    send(new_s, s, sizeof(s), 0);
   183	    for(i=0;i<end;i++){
   184	      print_profile(&profile_data_store[i], new_s);
   185	      //     printf("\n");
   186	  }
   187	  }else if(0 < nitems){
   188	    if(nitems > end) nitems = end;
   189	    snprintf(s, MAX_LINE_LEN, "%d", nitems);
   190	    send(new_s, s, sizeof(s), 0);
   191	    for(i=0;i<nitems;i++){
   192	      print_profile(&profile_data_store[i], new_s);
   193	      //     printf("\n");
   194	    }
   195	  }else if(nitems < 0){
   196	    end=end+nitems;
   197	    if(end< 0) end = 0;    
   198	    snprintf(s, MAX_LINE_LEN, "%d", end);
   199	    send(new_s, s, sizeof(s), 0);
   200	    for(i=end;i < profile_data_nitems;i++){
   201	      print_profile(&profile_data_store[i], new_s);
   202	      //      printf("\n");
   203	    }
   204	  }
   205	}
   206	
   207	/*[15]*/
   208	void cmd_read(char *filename, int new_s)
   209	{
   210	  char buffer[MAX_LINE_LEN + 1];
   211	  int a,b;
   212	  FILE *fp;
   213	  char s[MAX_LINE_LEN + 1] = "\0";
   214	
   215	  a = profile_data_nitems;
   216	  fp = fopen(filename, "r");
   217	
   218	  if(fp == NULL) {
   219	    snprintf(s, MAX_LINE_LEN, "ファイルがありません,ファイル名を確認してください.\n");
   220	    send(new_s, s, sizeof(s), 0);
   221	    return;
   222	  }
   223	  while(get_line(fp ,buffer))
   224	    {
   225	    new_profile(&profile_data_store[profile_data_nitems], buffer);
   226	    profile_data_nitems++;
   227	    back = 1;
   228	    ditems = 1;
   229	    }	
   230	  b = profile_data_nitems;
   231	  fclose(fp);
   232	
   233	  ditems = b - a;
   234	  back = 1;
   235	  snprintf(s, MAX_LINE_LEN, "読み込みが完了しました.%%C等で確認してください.\n");
   236	  send(new_s, s, sizeof(s), 0);
   237	}
   238	
   239	/*[16]*/
   240	void fprint_profile_csv(int i, FILE *fp)
   241	{
   242	  fprintf(fp,"%d,", profile_data_store[i].id);
   243	  fprintf(fp,"%s,", profile_data_store[i].name);
   244	  fprintf(fp,"%04d-%02d-%02d,", profile_data_store[i].birthday.y
   245		  ,profile_data_store[i].birthday.m, profile_data_store[i].birthday.d);
   246	  fprintf(fp,"%s,", profile_data_store[i].home);
   247	  fprintf(fp,"%s\n", profile_data_store[i].comment);
   248	}
   249	
   250	/*[17]*/
   251	void cmd_write(char *filename, int new_s)
   252	{
   253	  int i;
   254	  FILE *fp;
   255	  char *file = "writefile.csv";
   256	  char s[MAX_LINE_LEN + 1] = "\0";
   257	
   258	  if(*filename == 0) fp = fopen(file,"w");
   259	  else fp = fopen(filename, "w");
   260	
   261	  for(i = 0; i < profile_data_nitems; i++){
   262	    fprint_profile_csv(i,fp);
   263	  }
   264	
   265	  fclose(fp);
   266	
   267	  flag = 0;
   268	
   269	  snprintf(s,MAX_LINE_LEN,"書き込みが完了しました.ファイルを確認してください.\n");
   270	  send(new_s, s, sizeof(s), 0);
   271	}
   272	
   273	/*[18]*/
   274	void cmd_find(char *word, int new_s)
   275	{
   276	  int i;
   277	  int count = 0;
   278	  struct profile *p;
   279	  char id[MAX_ID_LEN+1];
   280	  char date[MAX_BIRTH_LEN+1];
   281	  char s[MAX_LINE_LEN +1] = {'\0'};
   282	
   283	  for(i=0;i<profile_data_nitems;i++){
   284	    p = &profile_data_store[i];
   285	    sprintf(id, "%d", p->id);
   286	    if(strcmp(id, word) == 0 ||
   287	       strcmp(p->name, word) == 0 ||
   288	       strcmp(date_to_string(date, &(p->birthday)), word) == 0 ||
   289	       strcmp(p->home, word) == 0 ||
   290	       strcmp(p->comment, word) == 0
   291	       ){
   292	      count++;
   293	    }
   294	  }
   295	  snprintf(s, MAX_LINE_LEN, "%d", count);
   296	  send(new_s, s, sizeof(s), 0);
   297	
   298	  for(i=0;i<profile_data_nitems;i++){
   299	    p = &profile_data_store[i];
   300	    sprintf(id, "%d", p->id);
   301	    if(strcmp(id, word) == 0 ||
   302	       strcmp(p->name, word) == 0 ||
   303	       strcmp(date_to_string(date, &(p->birthday)), word) == 0 ||
   304	       strcmp(p->home, word) == 0 ||
   305	       strcmp(p->comment, word) == 0
   306	       ){
   307	      snprintf(s, MAX_LINE_LEN,
   308		       "Id    : %d\nName  : %s\nBirth : %s\nAddr  : %s\nCom.  : %s\n", 
   309		       p->id, p->name, date_to_string(date, &p->birthday), p->home, p->comment);
   310	      send(new_s, s, sizeof(s), 0);
   311	    }
   312	  }
   313	}
   314	
   315	/*[19]*/
   316	void swap(struct profile *a, struct profile *b)
   317	{
   318	  struct profile tmp;
   319	  
   320	  tmp = *a;
   321	  *a = *b;
   322	  *b = tmp;
   323	}
   324	
   325	/*[20]*/
   326	int compare_date(struct date *d1, struct date *d2)
   327	{
   328	  if (d1->y != d2->y) return d1->y - d2->y;
   329	  if (d1->m != d2->m) return d1->m - d2->m;
   330	  return d1->d - d2->d;
   331	}
   332	
   333	/*[21]*/
   334	int profile_compare(struct profile *p1, struct profile *p2, int column)
   335	{
   336	  switch (column){
   337	  case 1:
   338	    return p1->id - p2->id; break;
   339	  case 2:
   340	    return strcmp(p1->name,p2->name); break;
   341	  case 3:
   342	    return compare_date(&(p1->birthday),&(p2->birthday)); break;
   343	  case 4:
   344	    return strcmp(p1->home,p2->home); break;
   345	  case 5:
   346	    return strcmp(p1->comment,p2->comment); break;
   347	  }
   348	  return 0; 
   349	}
   350	
   351	
   352	/*[22]*/
   353	void cmd_sort(int param, int new_s)
   354	{
   355	  int i, j;
   356	  char s[MAX_LINE_LEN + 1] = {'\0'};
   357	  struct profile *p;
   358	
   359	  if(0< param && param <6){
   360	  for (i = 0; i < profile_data_nitems -1; i++) {
   361	    for (j = 0; j < profile_data_nitems -1; j++) {
   362	      p = &profile_data_store[j];
   363	      
   364	      if (profile_compare(p, p+1, param) > 0)
   365		swap(p, p+1);
   366	    }
   367	  }
   368	  back = 0;
   369	  snprintf(s, MAX_LINE_LEN, "ソートが完了しました.%%Pなどで確認してください.\n");
   370	  send(new_s, s, sizeof(s), 0);
   371	  }else{
   372	    snprintf(s, MAX_LINE_LEN, "有効な引数は1~5です.正しい引数を入力してください.\n");
   373	    send(new_s, s, sizeof(s), 0);
   374	  }
   375	}
   376	
   377	/*[23]*/
   378	void ndelete(int nitems)
   379	{
   380	  int i;
   381	  for(i=0;i<nitems;i++){
   382	    free(profile_data_store[profile_data_nitems-1].comment);
   383	    profile_data_nitems--;
   384	  }
   385	}
   386	
   387	/*[24]*/
   388	void cmd_delete(int param, int new_s)
   389	{
   390	  int i;
   391	  FILE *fp;
   392	  char s[MAX_LINE_LEN + 1]={'\0'};
   393	  fp = fopen("backup.txt", "w");
   394	  mark = 0;
   395	
   396	  if(param == 0){
   397	    fprint_profile_csv(profile_data_nitems-1,fp);
   398	    ndelete(1);
   399	  }
   400	  else if(param > 0 && param< profile_data_nitems + 1){
   401	    for(i=0;i<param;i++)
   402	      fprint_profile_csv(profile_data_nitems-param+i,fp);
   403	    ndelete(param);
   404	  }
   405	  else if(param < 0 && -profile_data_nitems -1 <  param){
   406	    param = -param;
   407	    fprint_profile_csv(param-1,fp);
   408	    for(i=0;i<(profile_data_nitems -param);i++){
   409	      swap(&profile_data_store[param-1+i]
   410		   ,&profile_data_store[param+i]);
   411	    }
   412	    ndelete(1);
   413	    mark = param;
   414	  } else {
   415	    snprintf(s, MAX_LINE_LEN ,"保存件数は%d件です\n正しい引数を入力してください\n"
   416		    ,profile_data_nitems);
   417	    send(new_s, s, sizeof(s), 0);
   418	    return;
   419	  }
   420	  fclose(fp);
   421	  back = 2;
   422	  snprintf(s, MAX_LINE_LEN ,"delete ok\n");
   423	  send(new_s, s, sizeof(s), 0);
   424	}
   425	
   426	/*[25]*/
   427	void cmd_add(int param, int new_s)
   428	{
   429	  int i;
   430	  //  char line[MAX_LINE_LEN+1];
   431	  char s[MAX_LINE_LEN +1] = {'\0'};
   432	  struct profile *p;
   433	  mark = -param;
   434	
   435	  if(0 < param && param< profile_data_nitems){
   436	    snprintf(s, MAX_LINE_LEN, "CSV形式で名簿データを入力してください.\n");
   437	    send(new_s, s, sizeof(s), 0);
   438	
   439	    //    get_line(stdin,line);
   440	    bzero(&s, sizeof(s));
   441	    recv(new_s, s, sizeof(s), 0);
   442	     parse_line(s, new_s); 
   443	
   444	
   445	    p = &profile_data_store[profile_data_nitems-1];
   446	
   447	    for(i=0;i<(profile_data_nitems - param); i++){
   448	      swap(p-i-1,p-i);
   449	    }
   450	
   451	    back = 3;
   452	    /* snprintf(s, MAX_LINE_LEN, "登録が完了しました.\n"); */
   453	    /* send(new_s, s, sizeof(s), 0); */
   454	
   455	  }else if(param == profile_data_nitems){
   456	
   457	    snprintf(s, MAX_LINE_LEN, "CSV形式で名簿データを入力してください.\n");
   458	    send(new_s, s, sizeof(s), 0);
   459	
   460	    //   get_line(stdin,line);
   461	    bzero(&s, sizeof(s));
   462	    recv(new_s, s, sizeof(s), 0);
   463	    parse_line(s, new_s); 
   464	
   465	    p = &profile_data_store[profile_data_nitems-1];
   466	
   467	    swap(p-1,p);
   468	
   469	    back = 3;
   470	    /* snprintf(s, MAX_LINE_LEN, "登録が完了しました.\n"); */
   471	    /* send(new_s, s, sizeof(s), 0); */
   472	
   473	  }else{
   474	    snprintf(s, MAX_LINE_LEN, "登録件数は%d件です.正しい引数を入力してください.\n",profile_data_nitems);
   475	    send(new_s, s, sizeof(s), 0);
   476	  }
   477	}
   478	
   479	/*[26]*/
   480	void cmd_back(int new_s)
   481	{
   482	  int i;
   483	  char s[MAX_LINE_LEN +1]={'\0'};
   484	  struct profile *p;
   485	
   486	  switch(back){
   487	
   488	  case 0:
   489	    snprintf(s, MAX_LINE_LEN,"%%Bコマンドは,%%R,%%Aコマンド実行後しか使用できません.\n");
   490	    send(new_s, s,sizeof(s), 0);
   491	    break;
   492	    
   493	  case 1:
   494	    ndelete(ditems); 
   495	    snprintf(s, MAX_LINE_LEN,"%%Rコマンド実行前の状態に戻りました.\n"); 
   496	    send(new_s, s,sizeof(s), 0);
   497	    break;
   498	
   499	  case 2:
   500	    /* cmd_read("backup.txt", new_s); */
   501	    /* p = &profile_data_store[profile_data_nitems-1]; */
   502	    /* if(mark == 0){ */
   503	    /*   for(i=0;i<profile_data_nitems-mark;i++) */
   504	    /* 	swap(p-1-i,p-i); */
   505	    /* } */
   506	    snprintf(s, MAX_LINE_LEN,"%%Dコマンド実行前の状態に戻りました.\n");
   507	    send(new_s, s,sizeof(s), 0);
   508	    break;
   509	    
   510	  case 3:
   511	    cmd_delete(mark, new_s); 
   512	    // snprintf(s,MAX_LINE_LEN,"%%Aコマンド実行前の状態に戻りました.\n");
   513	    // send(new_s, s,sizeof(s), 0);
   514	    break;
   515	   
   516	  }
   517	
   518	  mark=0;
   519	  back=0;
   520	}
   521	
   522	/*[27]*/
   523	void cmd_man(int new_s)
   524	{
   525	  char s[MAX_LINE_LEN + 1] = {'\0'};
   526	  snprintf(s, MAX_LINE_LEN, "\nこのプログラムは標準入力から「ID,氏名,年月日,住所,備考」からなるコンマ区切り形式(CSV形式)の名簿データを受け付けて,それらを名簿中に登録する名簿管理プログラムである.\n下記では,%%で始まる各コマンド入力の仕様を説明している.\n\n%%Q      |終了(Quit)\n%%C      |登録件数などの表示(Check)\n%%P n    |先頭からn件表示(Print)\n%%R file |fileから読み込み(Read)\n%%W file |fileへ書き出し(Write)\n%%F word |wordを検索(Find)\n%%S n    |データをn番目の項目で整列(Sort)\n%%D n    |データをn件削除(Delete)\n%%A n    |n番目にデータを登録(Add)\n%%B      |直前の状態に戻る(Back)\n%%M      |各コマンドの仕様(Manual)\n\n");
   527	  send(new_s, s, sizeof(s), 0);
   528	}
   529	
   530	/*[28]*/
   531	void exec_command(char cmd, char *param, int new_s) //全てのコマンドにnew_sわたす いまはcだけ
   532	{
   533	  char s[MAX_LINE_LEN + 1] = {'\0'};
   534	  switch (cmd) {
   535	  case 'Q': cmd_quit(param, new_s); break; //-----------------[10] -aコマンド以外ok
   536	  case 'C': cmd_check(new_s); break; //-----------------------[11] ok
   537	  case 'P': cmd_print(atoi(param), new_s); break; //----------[14] ok
   538	  case 'R': cmd_read(param, new_s); break; //-----------------[15] ok
   539	  case 'W': cmd_write(param, new_s); break; //----------------[17] ok
   540	  case 'F': cmd_find(param, new_s); break; //-----------------[18] ok
   541	  case 'S': cmd_sort(atoi(param), new_s); break; //-----------[22] ok
   542	  case 'D': cmd_delete(atoi(param), new_s); break; //---------[24] ok 
   543	  case 'A': cmd_add(atoi(param), new_s); break; //------------[25] ok
   544	  case 'B': cmd_back(new_s); break; //------------------------[26] case2(D)以外ok
   545	  case 'M': cmd_man(new_s); break; //-------------------------[27] ok
   546	  default:
   547	    snprintf(s, MAX_LINE_LEN, "%cは登録されていないコマンドです. %%Mなどでコマンドを確認してください.\n",  cmd);
   548	    send(new_s, s, sizeof(s), 0);
   549	    break;
   550	  }
   551	}
   552	
   553	
   554	
   555	/*[29]*/
   556	void parse_line(char *line, int new_s)
   557	{
   558	  char s[MAX_LINE_LEN +1] = {'\0'};
   559	  if(line[0] == '%') {
   560	    exec_command(line[1], &line[3], new_s);
   561	  } else {
   562	    new_profile(&profile_data_store[profile_data_nitems], line);
   563	    profile_data_nitems++;
   564	    back = 1;
   565	    ditems = 1;
   566	    snprintf(s,MAX_LINE_LEN, "New Data added\n");
   567	    send(new_s, s, sizeof(s), 0);
   568	  }/* else if (new_profile(&profile_data_store[profile_data_nitems], line)!=NULL){
   569	      profile_data_nitems++;
   570	      back = 1;
   571	      ditems = 1;
   572	      
   573	      send(new_s, s, sizeof(s), 0);
   574	      } else {
   575	      fprintf(stderr,"入力形式が違います.\n");
   576	      }*/
   577	}
\end{verbatim}

\newpage

\subsubsection{meibo-client.c}
\begin{verbatim}
     1	#include <sys/fcntl.h>
     2	#include <sys/types.h>
     3	#include <sys/socket.h>
     4	#include <sys/stat.h>
     5	#include <netinet/in.h>
     6	#include <netdb.h>
     7	#include <stdio.h>
     8	#include <string.h>
     9	#include <strings.h>
    10	#include <stdlib.h>
    11	#include <unistd.h> //close
    12	#include <fcntl.h>
    13	
    14	#define PORT_NO 10016
    15	#define MAX_LINE_LEN 1024
    16	
    17	int subst(char *str, char c1, char c2)
    18	{
    19	  int n = 0;
    20	
    21	  while (*str) {
    22	    if (*str == c1) {
    23	      *str = c2;
    24	      n++;
    25	    }
    26	    str++;
    27	  }
    28	  return n;
    29	}
    30	
    31	int get_line(FILE *fp,char *line)
    32	{
    33	  if (fgets(line, MAX_LINE_LEN + 1, fp) == NULL)
    34	    return 0;
    35	
    36	  subst(line, '\n','\0');
    37	
    38	  return 1;
    39	}
    40	
    41	int main(int argc, char *argv[]){
    42	
    43	  /*通信相手のIPアドレスの取得*/
    44	
    45	  struct hostent* hostname;
    46	  if(argv < 0){
    47	    printf("Error : arguments number\n");
    48	  }
    49	
    50	  hostname = gethostbyname(argv[1]);
    51	  if(hostname == NULL){
    52	    printf("Error : hostname is NULL\n");
    53	  }
    54	
    55	  /*ソケットを作成する*/
    56	
    57	  int sockfd;
    58	  sockfd = socket(AF_INET, SOCK_STREAM, 0);
    59	  if(sockfd < 0){
    60	    printf("Error : can't make socket\n");
    61	    return(-1);
    62	  }
    63	
    64	  /*コネクションを確立する*/
    65	
    66	  struct sockaddr_in client_addr;
    67	
    68	  // memset((char*)&client_addr.sin_addr, 0, sizeof(client_addr.sin_addr));
    69	
    70	  memset((char*)&client_addr, 0, sizeof(client_addr));
    71	
    72	  client_addr.sin_family = hostname -> h_addrtype;
    73	  memcpy((char*)&client_addr.sin_addr, (char*)hostname -> h_addr, hostname -> h_length);
    74	  client_addr.sin_port = htons(PORT_NO);
    75	
    76	  if(connect(sockfd, (struct sockaddr *)&client_addr, sizeof(client_addr)) < 0){
    77	    printf("Error : can't connect\n");
    78	    return(-1);
    79	  }
    80	
    81	  /*メッセージを送信する*/
    82	
    83	  while(1){
    84	    printf("クライアントの入力待ち\n");
    85	
    86	    char line[MAX_LINE_LEN + 1];
    87	    get_line(stdin, line);
    88	    int check;
    89	    check = send(sockfd, line, sizeof(line), 0);
    90	
    91	    if(check < 0){
    92	      printf("Error : can't send\n");
    93	      return(-1);
    94	    };
    95	
    96	    /*メッセージを受信する*/
    97	    char kekka[MAX_LINE_LEN + 1];
    98	    if((line[0]=='%' && line[1]=='P') || (line[0]=='%' && line[1]=='F')){
    99	      bzero(&kekka, sizeof(kekka));
   100	      recv(sockfd, kekka, sizeof(kekka), 0); //回数を受け取る
   101	      int times;
   102	      int l;
   103	      times = atoi(kekka);
   104	      for(l=0; l<times; l++){
   105		bzero(&kekka, sizeof(kekka));
   106		recv(sockfd, kekka, sizeof(kekka), 0);
   107		printf("%s\n", kekka);
   108	      }
   109	
   110	    }else if(line[0]=='%' && line[1]=='A'){
   111		bzero(&kekka, sizeof(kekka));
   112		recv(sockfd, kekka, sizeof(kekka), 0);
   113		printf("%s\n", kekka);
   114	
   115		bzero(&kekka, sizeof(kekka));
   116		get_line(stdin,kekka);
   117		send(sockfd, kekka, sizeof(kekka), 0);
   118	
   119		bzero(&kekka, sizeof(kekka));
   120		recv(sockfd, kekka, sizeof(kekka), 0);
   121		printf("%s\n", kekka);
   122	
   123	
   124	    }else if(line[0]=='%' && line[1]=='B'){
   125	      	bzero(&kekka, sizeof(kekka));
   126		recv(sockfd, kekka, sizeof(kekka), 0);
   127		//	printf("%s\n", kekka);
   128		printf("Undo!\n");
   129	
   130	    }else if(line[0]=='%'){
   131	      bzero(&kekka, sizeof(kekka));
   132	      recv(sockfd, kekka, sizeof(kekka), 0);
   133	      printf("%s\n", kekka);
   134	      if(line[1]=='Q'){
   135		//	  exit(0);
   136		return 0;
   137	      }
   138	    }else{
   139	      bzero(&kekka, sizeof(kekka));
   140	      recv(sockfd, kekka, sizeof(kekka), 0);
   141	      printf("%s\n", kekka);
   142	    }
   143	    bzero(&line, sizeof(line));
   144	    /* // while(1){ */
   145	    /*   char kekka[MAX_LINE_LEN + 1]; */
   146	    /*   if(recv(sockfd, kekka, sizeof(kekka), 0) < 0){ */
   147	    /* 	printf("Error : can't recv\n"); */
   148	    /* 	return(-1); */
   149	    /*   }; */
   150	
   151	    /*   printf("%s\n", kekka); */
   152	
   153	  }
   154	
   155	  /*ソケットを削除する*/
   156	  if(close(sockfd) < 0){
   157	    printf("Error : can't close\n");
   158	    return(-1);
   159	  }
   160	  return 0;
   161	}
\end{verbatim}

\newpage

\subsubsection{meibo-server.c}
\begin{verbatim}
     1	#include <stdio.h>
     2	#include <string.h>
     3	#include <stdlib.h>
     4	#include <sys/fcntl.h>
     5	#include <sys/types.h>
     6	#include <sys/socket.h>
     7	#include <sys/stat.h>
     8	#include <netinet/in.h>
     9	#include <netdb.h>
    10	#include <string.h>
    11	#include <strings.h>
    12	#include <unistd.h> //close
    13	#include <fcntl.h>
    14	#include <errno.h>
    15	
    16	#define PORT_NO 10016
    17	#define MAX_LINE_LEN 1024
    18	
    19	void parse_line(char *line, int new_s);
    20	
    21	int main(){
    22	
    23	  /*ソケットを作成する*/
    24	
    25	  int sockfd;
    26	  int yes = 1;
    27	  sockfd = socket(AF_INET, SOCK_STREAM, 0);
    28	  if(sockfd < 0){
    29	    printf("Error : can't make socket\n");
    30	    return(-1);
    31	  }
    32	
    33	  /* SO_REUSEADDR をつける*/
    34	
    35	  int ret;
    36	  ret = setsockopt(sockfd, SOL_SOCKET, SO_REUSEADDR,(const char *)&yes, sizeof(yes));
    37	  if(ret < 0){
    38	    printf("Error : can't opt");
    39	    return 0;
    40	  }
    41	
    42	  /*ソケットに名前をつける*/
    43	
    44	  struct sockaddr_in reader_addr;
    45	
    46	  memset((char*)&reader_addr, 0, sizeof(reader_addr));
    47	
    48	  reader_addr.sin_family = AF_INET; /*インターネットドメイン*/
    49	  reader_addr.sin_addr.s_addr = htonl(INADDR_ANY); /*任意のIPアドレスを受付*/
    50	  reader_addr.sin_port = htons(PORT_NO); /*接続待ちのポート番号を設定*/
    51	
    52	  int name;
    53	  name = bind(sockfd, (struct sockaddr *)&reader_addr, sizeof(reader_addr));
    54	  if(name < 0){
    55	    perror("bind");
    56	    printf("Error : bind\n");
    57	    return(-1);
    58	  }
    59	
    60	  /*接続要求を待つ*/
    61	
    62	  int wait;
    63	  wait = listen(sockfd,5);
    64	  if(wait < 0){
    65	    printf("Error : listen\n");
    66	    close(sockfd);
    67	    return(-1);
    68	  }
    69	
    70	  /*接続要求を受け付ける*/
    71	  int new_s;
    72	    while(1){
    73	
    74	  struct sockaddr_in client;
    75	  
    76	  socklen_t len = sizeof(client);
    77	  new_s = accept(sockfd, (struct sockaddr *)&client, &len);
    78	  if(new_s < 0){
    79	    printf("Error : accept\n");
    80	    return(-1);
    81	  }
    82	
    83	  /*メッセージを受信する*/
    84	
    85	  while(1){
    86	    char buf[MAX_LINE_LEN + 1];
    87	    printf("クライアントの入力待ち \n");
    88	    recv(new_s, buf, sizeof(buf), 0);
    89	    printf("入力 %s\n",buf);
    90	    if(buf[0]=='%' && buf[1]=='Q'){
    91	    parse_line(buf, new_s);
    92	    printf("処理終了\n\n");
    93	    break;
    94	    }else{
    95	      printf("サーバの処理開始 \n");
    96	
    97	      parse_line(buf, new_s);
    98	
    99	      printf("入力 after parse_line(): %s\n", buf);
   100	      bzero(&buf, sizeof(buf));
   101	      printf("処理終了\n\n");
   102	    }
   103	  }
   104	
   105	
   106	    /* /\*メッセージの送信*\/ */
   107	    /* memset(buf,0,sizeof(s)); */
   108	    /* send(new_s, buf, sizeof(buf), 0); */
   109	
   110	  }
   111	
   112	  /*ソケットの削除*/
   113	
   114	  close(new_s);
   115	
   116	  //  } dekai while
   117	
   118	  return 0;
   119	
   120	}
\end{verbatim}

\end{document}



